\documentclass[
% opciók nélkül: egyoldalas nyomtatás, elektronikus verzió
% twoside,     % kétoldalas nyomtatás
% tocnopagenum,% oldalszámozás a tartalomjegyzék után kezdődik
]{thesis-ekf}
\usepackage[T1]{fontenc}
\PassOptionsToPackage{defaults=hu-min}{magyar.ldf}
\usepackage[magyar]{babel}
\usepackage{mathtools,amssymb,amsthm,pdfpages}
\footnotestyle{rule=fourth}

\newtheorem{tetel}{Tétel}[chapter]
\theoremstyle{definition}
\newtheorem{definicio}[tetel]{Definíció}
\theoremstyle{remark}
\newtheorem{megjegyzes}[tetel]{Megjegyzés}

\begin{document}
		\institute{Matematikai és Informatikai Intézet}
	\title{Alkalmazásfejlesztés WEB alapokon}
	\author{Géczi Bálint\\Programtervező informatikus}
	\supervisor{Balla Tamás\\tanársegéd}
	\city{Eger}
	\date{2023}
	\maketitle
	\tableofcontents
	
	\chapter{Bevezetés}
	\addcontentsline{toc}{chapter}{Bevezetés}
	Napjainkban az internet használatának elterjedésével bővültek a lehetőségeink minden szempont tekintetében. A világháló számos lehetőséget ad számunkra csekkek feladatásától kezdve az online rendelésekig vagy akár új emberek megismerését is sokkal inkább könnyebbé váltak mint az internet előtti időkben. 
	
	Előnyként sorolandó a mai világban az is, hogy szinte mindenkinek van internetelérése alkalmazható eszköze és ezáltal az előnyök élvezete is elérhető mindenki számára. Az informatikai cikkek térhódítása nem csak a kompjúterekre vonatkozik hanem más elektronikai eszközeinkre is mint mobiltelefonunkra és egyéb internetelérést biztosított eszközeinkre. Nem feltétlen szükséges ilyen eszközök birtokban sem lennünk hiszen netkávézókban, könyvtárakban is biztosítva van erre számos lehetőségünk.
	
	Az átlag emberek felhasználói felületeinek bővülése mellett a fejlesztőknek is rengeteg újdonsággal bővült fejlesztést segítő programjainak palletája. Fejlődött a webdizájn is és ennek következtében az oldalak is egyre felhasználóbarátabbá váltak.
	
	Kezdetben az weboldalak, webalkalmazások fejlesztése HTML, CSS és Javascript webfejlesztési technikákból álltak azóta ezeknek a technológiáknak a körei is jócskán bővültek mint például CSS3, WebGL, HTML 5, Java, React JS, Angular JS, PHP. A szakdolgozatom során a Laravel által biztosított fejlesztői környezetet fogom használni.
	
	\section{Szakdolgozatom célja}
	A weboldal célja, hogy a látogató könnyen tudjon magának személyi edzőt keresni és ezáltal a fejlődése egy újabb szintre léphet mind a helyes táplálkozás, mind az edzés terén. Az oldal biztosítja azt a felületet, ahol az edzők, akik felregisztrálják magukat az oldalra és ott megtalálhatók lesznek. A regisztráció persze nem csak az edzők számára elérhető, hanem azok számára is akik éppen pont egy edzőt vagy egy edzőtársat keresnek maguk mellé. A fő oldalon belépés előtt elérhető számos cikk az egészséges táplálkozással kapcsolatban illetve a helyes edzés kivitelezésével kapcsolatban is.
	
	Fontosnak tartottam mindig is az egészséges életvitelt és ez az oldal hozzájuthat ahhoz, hogy azok akik éppen elakarják kezdeni a sportot legyen egy alapjuk mind tudás szinten vagy akár edzőtárs szempontjából is. Azonban azok számára is kiváló az oldal akik nem kezdő szinten állnak mert mindenki számára pozitív hatásai lehetnek az oldal igénybevétele után.
	
	Az oldal megvalósításának célja mind az embereknek való segítség a témához kapcsolódóan és emellett képességeim fejlesztése a webfejlesztés terén.
	
	\section{Piackutatás}
	Mielőtt nekiálltam a szakdolgozatom elkészítésének piackutatást\footnote{Egy olyan közgazdaságtani vizsgálat, amely egy-egy konkrét üzleti cél érdekében végez adatgyűjtést illetve elemzést.} végeztem azzal kapcsolatban, hogy létezik e már ilyen weboldal és ha igen ők mit kínálnak. Az eredménye a piackutatásnak végül az lett, hogy létezik már ilyen weboldal de nem sok ember használja és kevésbé ismert, emellett funkcióiban eltérő az én projektemhez viszonyítva.
	
	Az oldal tehát számos új funkciókkal fogja használóit ellátni, amit még eddig más weboldalakon nem tapasztalhattak meg.
	
	\chapter{Felhasznált technológiák}
	\section{Szakasz címe}
	
	\chapter{Felhasználói körök}
	\section{Engedélyek/felület a látogatóknak, akik még nem regisztráltak}
	Mivel még nem történt meg a regisztráció számos funkció nem elérhető számukra, de persze regisztráció nélkül is lehet élvezni az oldal számos előnyét. Egyik ilyen előny közé sorolnám az oldal cikkeinek olvasását a helyes és egészséges étrendről, illetve pár tanácsot az edzéssel kapcsolatban, illetve a cikkek mellett még megtudják nézni a magánedzőket akik fent vannak az oldalon viszont nincs lehetőségük felkeresni velük a kapcsolatot és nem tudnak időpontot foglalni náluk. Ha engedélyezik a hely jelenlegi helyadatok megadását akkor lehetőség nyílik rá, hogy azokat a magánedzőket előbbre vegye, akik közel vannak hozzájuk. A weboldalon lesz egy terem fül is ami a termeket is megjeleníti a közelben. 
	
	\section{Engedélyek/felület a regisztrált felhasználók számára}
	Egyértelműen a belépett felhasználók megkapják ugyanazokat az engedélyeket, amiket a nem belépet felhasználók is megkaptak. Időpontot foglalni a kiválasztott magánedzőkhöz, ezt segíti, hogy meg lesz jelenítve, hogy az edző mely napokon és órákban ér rá és ezek közül is mi nincsen még lefoglalva. Lehetőségük van társalogni velük a weboldalnak köszönhetően. Belépés után lehetőség lesz egy saját edzés terv beállítására, illetve egy haladási napló vezetésére is. A haladási naplóba be lehet írni a jelenlegi maximum kilogrammot, amivel tudja a gyakorlatot csinálni, illetve a maximum ismétlés és sorozat számot is. Fogja tudni vezetni a saját súlyának haladását is ha szeretné, hogy lássa mégis, hogy halad. Lehetőség lesz még saját magát is felvinni a rendszerbe az embernek arra az esetre, ha esetleg egy edző társat szeretne keresni vagy lehetősége van nem csak egy hanem akár több ember keresésére is.
	
	\section{Engedélyek/felület, edzők számára}
	Ha valaki edzőként lép be jogosultságot kap, arra, hogy meghirdesse magát, írjon egy bemutatkozó szöveget, képeket töltsön fel az oldalára, ami alapján eltudják dönteni azok az emberek, akik menni szeretnének hozzá, hogy az edző milyen formában van. Betudja állítani, hogy mely időpontok a jók számára, amiket a felhasználók le tudnak foglalni nála. Illetve megtudja adni árazását mind az magánedzés szempontjából és mind a helyes étrend összeállításért elkért pénzt, persze csak vállal ilyet.
	
	\section{Engedélyek/felület, adminisztrátor}
	Adminisztrátorként való belépés után egyik fülön lehetőség van megnézni, hogy kik jelentkeztek eddig az oldalra magánedzőnek, illetve lehetőségük van a hírek/cikkek szerkesztésére.
	
	
	\chapter*{Összegzés}
	\addcontentsline{toc}{chapter}{Összegzés}
	Még nincsen.
	
	\begin{thebibliography}{2}
		\addcontentsline{toc}{chapter}{\bibname}
		\bibitem{pelda}
		\textsc{Pelda István}: \emph{Peldaoldal}, Debreceni Egyetem, Debrecen, 2004.
	\end{thebibliography}
	
	% Aláírt, szkennelt nyilatkozat beillesztése a szakdolgozat végére
	% \includepdf{nyilatkozat.pdf}
\end{document}